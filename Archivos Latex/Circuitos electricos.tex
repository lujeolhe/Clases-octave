%%%%%%%%%%%%%%%%%%%%%%%%%%%%%%%%%%%%%%%%%%%%%%%%%%
%Proyecto: Ejemplo de latex
%Colaboradores: Ljoh
%Fecha: 26 sep 2020
%%%%%%%%%%%%%%%%%%%%%%%%%%%%%%%%%%%%%%%%%%%%%%%%%%%

%================================================================
%				Tipo de documeto
%================================================================
\documentclass[12pt]{article}
%10pt,12pt twocolumns
%================================================================
%					Preambulo
%================================================================
\usepackage{lmodern}
\usepackage[utf8]{inputenc}
\usepackage[T1]{fontenc}
\usepackage[spanish,activeacute]{babel}

\usepackage{multicol}
\usepackage{enumerate}
\usepackage{enumitem}
\usepackage{booktabs}
\usepackage{tabularx, makecell}


\usepackage{lipsum}
\usepackage{float}

\usepackage{mathtools}
\usepackage{amssymb, amsmath, amsbsy} % simbolitos
%\usepackage{upgreek} % para poner letras griegas sin cursiva
\usepackage{cancel} % para tachar
\usepackage{mathdots} % para el comando \iddots
\usepackage{mathrsfs} % para formato de letra
\usepackage{stackrel} % para el comando \stackbin
\usepackage{multirow} % para las tablas
%\usepackage[fleqn]{amsmath}
\usepackage{nccmath}
\usepackage{multicol} %Texto en multiples columnas

\usepackage{graphicx} % figuras
\usepackage{subfigure} % subfiguras

%Circuitos
\usepackage{tikz,pgfplots}
\usepackage[europeanresistors,americaninductors]{circuitikz}

\usetikzlibrary{calc}
\usetikzlibrary{patterns}

\ctikzset{bipoles/thickness=1}%Grosor de los elementos pasivos
\ctikzset{bipoles/length=1.2cm}%Longitud de los elementos pasivos
\tikzstyle{every node}=[font=\normalsize] %tamaño de etiquetas
\tikzstyle{every path}=[line width=1.25pt, line cap=round, line join=round] %caracteristicas de la linea union


\usepackage{wasysym}


\usepackage{xcolor}

%================================================================
%					Comandos
%================================================================
\definecolor{ColorRespuesta}{RGB}{12,79,182}
\newcommand{\Respuesta}[1]{\textcolor{ColorRespuesta}{#1}}
\newcommand{\derivada}[2]{\displaystyle{\frac{d#1}{d#2}}}
%================================================================
%					Margenes
%================================================================
\setlength{\textwidth}{170mm}%Ancho de Texto
\setlength{\textheight}{230mm}%Largo del Texto
\setlength{\oddsidemargin}{-5mm}%Margen de pagunas impares
\setlength{\evensidemargin}{5mm}%Margen de páginas pares
								%-para documentos tipo book-
\setlength{\topmargin}{-20mm}%Margeb Superior

%================================================================
%				  Datos del Autor
%================================================================
\title{Circuitos electricos practica}
\author{Luis Pablo}
\date{\today}
%================================================================
%					Documento
%================================================================


\begin{document}

\maketitle

\begin{figure}[htb]%h: here, t: top al principio de la pagina, b: al final de la página, !
		\centering %alineación 
		%%Figura*********************************************************
							%Simbologia europea o ameriaca, escala de la figura
		\begin{circuitikz}[american,scale=1.5, transform shape]
		    \draw (0,0) to [short,-*](0,0); %punto de voltaje -
			\draw (0,2) to [short,*-](0,2); %punto de voltaje +
			\draw (0,2) to [sV=$V_{i}$](0,0); %Etiqueta voltaje
			%filtro LCL
			\draw (0,2) to [L,l=$L_{1}$,f=$i_{L_1}$](3.1,2); %Inductacia L1 y corriente i1
			%\draw (3.1,2) to [L,l=$L_{2}$,f=$i_{g}$](5.1,2); %Inductacia L2 y corriente ig
			%\draw (5.1,2) to [short,-*](5.6,2); %punto de voltaje -
			%\draw (5.6,0) to [short,*-](3.1,0); %punto de voltaje +
			\draw (3.1,0) to [C, l=$C_{f}$, f=$i_{C_{f}}$](3.1,2); %Inductacia L2 y corriente ig
			%\draw (0,0) to [short,-](3.1,0); %punto de voltaje +
			\draw  (3.1,0) to [R,l=$R_1$,f=$i_{R_{1}}$](0,0);
			%\draw (5.6,2) to [open,v=$V_{o}$](5.6,0); %Etiqueta voltaje

		\end{circuitikz}
		%%************************************************************Final Figura
\end{figure}

		\begin{figure}[htb]%h: here, t: top al principio de la pagina, b: al final de la página, !
		\centering %alineación 
		%%Figura*********************************************************
							%Simbologia europea o ameriaca, escala de la 
\begin{circuitikz}[european, voltage shift=0.5] 
	\draw (0,0)
	to[isourceC, l=$I_0$, v=$V_0$] (0,3)
 	to[short, -*, f=$I_0$] (2,3)
 	to[R=$R_1$, f>_=$i_1$] (2,0) -- (0,0);
 	\draw (2,3) -- (4,3)
 	to[R=$R_2$, f>_=$i_2$]
 	(4,0) to[short, -*] (2,0);
 \end{circuitikz}
 \bigskip
 \\Circuito en serie\\
 \bigskip
 \begin{circuitikz}[european,scale=1.5, transform shape] 
	\draw (0,0)to [short,*-o](0,0)
 	to[R=$R_1$] (2,0)
 	to[R=$R_2$] (4,0)
 	to[R=$R_3$] (6,0)
 	to [short,*-o](6,0);
 	
 \end{circuitikz}
  \bigskip
  \end{figure}
  
		\begin{figure}[htb]%h: here, t: top al principio de la pagina, b: al final de la página, !
		\centering %alineación 
		%%Figura*********************************************************
\bigskip		%Simbologia europea o ameriaca, escala de la 
Circuito en paralelo\\
\bigskip
 \begin{circuitikz}[european,scale=1.5, transform shape] 
	\draw (0,0)to [short,*-o](0,0)
 	 to [short,-*] (0.5,0)
 	 to[R=$R_2$] (2,0)
 	 to [short,-*] (2,0)
 	 to [short,*-o](2.5,0)
 	 (0.5,0)--(0.5,1)
 	 (0.5,0)--(0.5,-1)
 	 ;
 	 \draw (0.5,1)
 	 to[R=$R_1$] (2,1)
 	 (2,1)--(2,0);
 	 
 	 \draw (0.5,-1)
 	 to[R=$R_3$] (2,-1)
 	 (2,-1)--(2,0);
 	
 \end{circuitikz}
 \bigskip 

 Circuito serie-paralelo\\
 \bigskip
  \begin{circuitikz}[european,scale=1.5, transform shape] 
	\draw (0,0)to [short,*-o](0,0)
 	 to[R=$R_2$,-*] (2,0)
 	 (2,0)--(2,1)
 	 (2,0)--(2,-1);
 	 
 	 \draw (2,1)
 	 to[R=$R_1$] (4,1)
 	 to[short,-*](4,0)
 	 to[short,*-o](4.5,0) 	 ;
 	 
 	 \draw (2,-1)
 	 to[R=$R_3$] (4,-1)
 	 (4,-1)--(4,0);
 	
 \end{circuitikz}
 \bigskip
 \begin{circuitikz}[european,scale=1.5, transform shape] 
	\draw (0,0)to [short,*-o](0,0)
 	 to [short,-*] (0.5,0)
 	 (0.5,0)--(0.5,0.5)
 	 (0.5,0)--(0.5,-0.5);
 	 
 	 \draw (0.5,0.5)
 	 to[R=$R_1$] (2,0.5)
 	 to[R=$R_2$] (3,0.5)
 	 (3,0.5)--(3.5,0.5)
 	 (3.5,0.5)--(3.5,0);
 	 
 	 \draw (0.5,-0.5)
 	 to[R=$R_3$] (2,-0.5)
 	 to[R=$R_4$] (3,-0.5)
	 (3,-0.5)--(3.5,-0.5)
 	 (3.5,-0.5)to [short,-*](3.5,0)
 	 to[short,*-o](4,0);
 	
 \end{circuitikz}
 \bigskip
 
 Conexion estrella\\
 \bigskip
 \begin{circuitikz}[european,scale=1.5, transform shape] 
	\draw (0,2)to [short,*-o](0,2)
	to[R=$R_1$,-*] (0,0);
 	 \draw (-1.5,-1.5)
 	 to[short,*-o](-1.5,-1.5)
 	 to[R=$R_2$] (0,0);
 	 
 	 \draw (0,0)
 	 to[R=$R_3$,*-o] (1.5,-1.5);
 	
 \end{circuitikz}
 \end{figure}
 \begin{figure}[htb]%h: here, t: top al principio de la pagina, b: al final de la página, !
		\centering %alineación 
 Conexion triangulo\\
 \bigskip
 \begin{circuitikz}[european,scale=1.5, transform shape] 
	\draw (0,0)to [short,-*](0,0)
	to[R=$R_1$,-*] (2.5,0);
	
 	 \draw (0,0)
 	 to[R=$R_2$,-*] (1.25,2.5)
 	 to[R=$R_2$] (2.5,0);
 	
 \end{circuitikz}
 \bigskip
 \\Conexion en Uve\\
 \bigskip
 \begin{circuitikz}[european,scale=1.5, transform shape] 
	\draw (0,0)
	to [short,-*](0,0)
	to [short,*-o](0,2);
	
 	 \draw (0,0)
 	 to[R=$R_2$,*-o] (1.5,1.5);
 	 
 	 \draw (-1.5,1.5)
 	 to [short,*-o](-1.5,1.5)
 	 to[R=$R_1$,-*] (0,0);
 	
 \end{circuitikz}
 
 \bigskip
 Conexion en Zigzag\\
 \bigskip
 \begin{circuitikz}[european,scale=1.5, transform shape] 
	\draw (0,1.5)
	to [R=$R_5$](0,0)
	to [short,*-o](0,0)
	(0,1.5) --(1.5,2)
	(1.5,3.5)
	to [short,*-o](1.5,3.5)
	to [R=$R_3$](1.5,2)
	(0,0)--(-1.5,0)
	(0,0)--(1.5,0);
	
	\draw (-1.5,1.5)
	to [R=$R_4$](-1.5,0)
	(-1.5,1.5) --(0,2)
	(0,3.5)
	to [short,*-o](0,3.5)
	to [R=$R_2$](0,2);
	
	\draw (1.5,1.5)
	to [R=$R_6$](1.5,0)
	(1.5,1.5) --(-1.5,2)
	(-1.5,3.5)
	to [short,*-o](-1.5,3.5)
	to [R=$R_1$](-1.5,2);
	
	
 	
 \end{circuitikz}
 
		%%************************************************************Final Figura
		
	\end{figure}
	

\end{document}