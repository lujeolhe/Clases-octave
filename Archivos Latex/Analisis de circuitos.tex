%%%%%%%%%%%%%%%%%%%%%%%%%%%%%%%%%%%%%%%%%%%%%%%%%%
%Proyecto: Ejemplo de latex
%Colaboradores: Ljoh
%Fecha: 26 sep 2020
%%%%%%%%%%%%%%%%%%%%%%%%%%%%%%%%%%%%%%%%%%%%%%%%%%%

%================================================================
%				Tipo de documeto
%================================================================
\documentclass[12pt]{article}
%10pt,12pt twocolumns
%================================================================
%					Preambulo
%================================================================
\usepackage{lmodern}
\usepackage[utf8]{inputenc}
\usepackage[T1]{fontenc}
\usepackage[spanish,activeacute]{babel}

\usepackage{multicol}
\usepackage{enumerate}
\usepackage{enumitem}
\usepackage{booktabs}
\usepackage{tabularx, makecell}


\usepackage{lipsum}
\usepackage{float}

\usepackage{mathtools}
\usepackage{amssymb, amsmath, amsbsy} % simbolitos
%\usepackage{upgreek} % para poner letras griegas sin cursiva
\usepackage{cancel} % para tachar
\usepackage{mathdots} % para el comando \iddots
\usepackage{mathrsfs} % para formato de letra
\usepackage{stackrel} % para el comando \stackbin
\usepackage{multirow} % para las tablas
%\usepackage[fleqn]{amsmath}
\usepackage{nccmath}
\usepackage{multicol} %Texto en multiples columnas

\usepackage{graphicx} % figuras
\usepackage{subfigure} % subfiguras

%Circuitos
\usepackage{tikz,pgfplots}
\usepackage[europeanresistors,americaninductors]{circuitikz}

\usetikzlibrary{calc}
\usetikzlibrary{patterns}

\ctikzset{bipoles/thickness=1}%Grosor de los elementos pasivos
\ctikzset{bipoles/length=1.2cm}%Longitud de los elementos pasivos
\tikzstyle{every node}=[font=\normalsize] %tamaño de etiquetas
\tikzstyle{every path}=[line width=1.25pt, line cap=round, line join=round] %caracteristicas de la linea union

\usepackage{wasysym}

\usepackage{xcolor}
%================================================================
%					Comandos
%================================================================
\definecolor{ColorRespuesta}{RGB}{12,79,182}
\newcommand{\Respuesta}[1]{\textcolor{ColorRespuesta}{#1}}
\newcommand{\derivada}[2]{\displaystyle{\frac{d#1}{d#2}}}
\newcommand{\e}[1]{e^{#1}}
\newcommand{\integral}[4]{\displaystyle{\int_{#1}^{#2}{#3}{#4}}}
\newcommand{\escribir}[1]{\singlespacing #1 \singlespacing}
\newcommand{\fraccion}[2]{\displaystyle{\frac{#1}{#2}}}
\newcommand{\tam}[1]{\Large\raggedright #1}
\newcommand{\sub}[1]{\fcolorbox{black}{red}{#1}}
%================================================================
%					Margenes
%================================================================
\setlength{\textwidth}{170mm}%Ancho de Texto
\setlength{\textheight}{230mm}%Largo del Texto
\setlength{\oddsidemargin}{-5mm}%Margen de paginas impares
\setlength{\evensidemargin}{5mm}%Margen de páginas pares
								%-para documentos tipo book-
\setlength{\topmargin}{-30mm}%Margeb Superior

%================================================================
%				  Datos del Autor
%================================================================
\title{Analisis de circuitos electricos}
\author{Luis Pablo González Gálvez}

%================================================================
%					Documento
%================================================================
\begin{document}
\maketitle
%Indice de figuras:
	\listoffigures
	%%\tableofcontents
	\section{Circuitos Electricos}
	\subsection{Circuitos RC}
	\bigskip
%%Circuito 1
\begin{figure}[htb]%h: here, t: top al principio de la pagina, b: al final de la página, !
		\centering %alineación 
		%%Figura*********************************************************
							%Simbologia europea o ameriaca, escala de la figura
		\begin{circuitikz}[american,scale=2.5, transform shape]
		    \draw (0,0) to [short,-*](0,0); %punto de voltaje -
			\draw (0,2) to [short,*-](0,2); %punto de voltaje +
			\draw (0,2) to [sV=$V$](0,0); %Etiqueta voltaje
			%filtro LCL
			\draw (0,2) to [R,l=$R_{1}$,f=$i_{R_1}$](3.1,2); %Inductacia L1 y corriente i1
			%\draw (3.1,2) to [L,l=$L_{2}$,f=$i_{g}$](5.1,2); %Inductacia L2 y corriente ig
			%\draw (5.1,2) to [short,-*](5.6,2); %punto de voltaje -
			%\draw (5.6,0) to [short,*-](3.1,0); %punto de voltaje +
			\draw (3.1,2) to [C, l=$C$, f=$i_{C}$](3.1,0); %Inductacia L2 y corriente ig
			\draw (3.1,2)to[short,-](5.1,2);
			\draw  (5.1,2) to [R,l=$R_2$,f=$i_{R_{2}}$](5.1,0);
			\draw (5.1,0) to [short,-](0,0); 
		\end{circuitikz}
		\caption{RCSimple}
 		\label{RCSimple}
		%************************************************************Final Figura
\end{figure}
\bigskip


\tam{}Se tiene los siguientes valores de los componentes:\\

$V_d=5v$ \'o $V_a=5sin(2\pi60t) $\\
$R_1=100 \Omega, R_2=10 \Omega, C=20\mu f $\\
Se utiliza la ley de Kirchhorff de la corrientes.\\
$\sum I_e=\sum I_s$\\
$\Rightarrow I_{R_1}= I_C + I_{R_2}$\\
Donde:\\
\bigskip
$I_{R_1}=\fraccion{V_{R_1}}{R_1},I_C=C\derivada{V_c}{t} ,I_{R_2}=\fraccion{V_{R_2}}{R_2} $\\
\bigskip
Poner todo en funcion de $V_c$\\
Usando la ley de los Voltajes de Kirchhoff\\
$-V+V_{R_1}+V_C=0$\\
$V_{R_1}=V-V_c$\\
$V_{R_2}=V_c$
Por lo tanto:\\
\bigskip
$I_{R_1}=\fraccion{V-V_C}{R_1}, I_C=C\derivada{V_c}{t} , I_{R_2}=\fraccion{V_C}{R_2} $\\
\bigskip
Sustituimos en la ecuacion de la ley de las corrientes:\\
\bigskip
$\fraccion{V-V_C}{R_1}=C\derivada{V_c}{t} +\fraccion{V_C}{R_2} $\\
\bigskip
Se procede a quitar la C de la derivada:\\
\bigskip
$\fraccion{1}{C}(\fraccion{V-V_C}{R_1}=C\derivada{V_c}{t} +\fraccion{V_C}{R_2}) $\\
\bigskip
Procedemos a colocar bien la ecuacion diferencial: \\
\bigskip
$\derivada{V_c}{t} +\fraccion{V_C}{CR_2}- \fraccion{V}{CR_1}+\fraccion{V_C}{CR_1}=0 $\\
\bigskip
\bigskip
$\derivada{V_c}{t} +\fraccion{R_1+R_2}{CR_2R_1}V_c=\fraccion{V}{CR_1} $\\
\bigskip
Se procede a resolver la ecuacion diferencial:
Primero se resulve de forma Homogenia.\\
\bigskip
$\derivada{V_c}{t} +\fraccion{R_1+R_2}{CR_2R_1} V_c=0 $\\
\bigskip
Se considera $K=\fraccion{R_1+R_2}{CR_2R_1}$\\
\bigskip
Se utiliza el metodo de variables separables:\\
\bigskip
$\derivada{V_c}{t} =-K V_c $\\
\bigskip
$\int \fraccion{dV_c}{KV_c}=\integral{0}{t}{-1}{dt}$\\
\bigskip
$\fraccion{1}{K} ln(V_C)+ln(A)=-t$\\
\bigskip
$\fraccion{1}{K} ln(V_C*A)=-t$\\
\bigskip
$ln(V_C*A)=-tK$\\
\bigskip
$V_C*A=\e{-tK}$\\
La respuesta es:\\
\bigskip
\sub{$V_C=A\e{-tK}$}\\
\bigskip
Para la respuesta particular de la ecuacion diferencial:\\
$V_C=b$\\
$Kb=\fraccion{V_C}{CR_1}$\\
\bigskip
$b=\fraccion{V_C}{CR_1K}=\fraccion{V_CR_2}{R_1+R_2}$\\
\bigskip
La ecuacion completa es:\\
\bigskip
\sub{$V_C=A\e{-tK}+\fraccion{V_CR_2}{R_1+R_2}$}\\
\bigskip
Para el Voltaje en corriente alterna el Voltaje es:\\
$V_a=5sin(2\pi60t) $\\
Como ya se tiene la solucion Homogenia se busca la solucion particular
$V_C=Asin(\omega t)+Bcos(\omega t)$\\
\bigskip
$\derivada{V_c}{t}=A\omega cos(\omega t)-B\omega sin(\omega t)$\\
\bigskip
Se sustituye en la ecuacion (1)\\
\bigskip
$\derivada{V_c}{t}+KV_C=5sin(\omega t)$\\
\bigskip
$A\omega cos(\omega t)-B\omega sin(\omega t)+AKsin(\omega t)+BKcos(\omega t)=5sin(\omega t)$\\
\bigskip
Ahora se separan por los que tienen las mismas funciones trigonometricas\\
$-B\omega sin(\omega t)+AKsin(\omega t)=5sin(\omega t)$\\
$A\omega cos(\omega t)+BKcos(\omega t)=0$\\
\bigskip
Se factorizan las funciones y queda esto:\\
$-B\omega+AK=5$\\
$BK+A\omega=0$\\
\bigskip
Se quitan las K y $\omega$ de las B para poder resolverlo por sumas de sistemas de ecuaciones\\
\bigskip
$-B+A\fraccion{K}{\omega}=\fraccion{5}{\omega}$\\
\bigskip
$B+A\fraccion{\omega}{K}=0$\\
\bigskip
$A(\fraccion{K}{\omega}+\fraccion{\omega}{K})=\fraccion{5}{\omega}$\\
\bigskip
$A(\fraccion{K^2+\omega ^2}{\omega K })=\fraccion{5}{\omega}$\\
\bigskip
$A=\fraccion{5K}{K^2+\omega ^2}$\\
\bigskip
Sustituimos A en una de las ecuaciones anteriores.\\
$B=-A\fraccion{\omega}{K}$\\
\bigskip
$B=-\fraccion{5K\omega}{K(K^2+\omega ^2)}$\\
\bigskip
$B=-\fraccion{5\omega}{K^2+\omega ^2}$\\
\bigskip
Por lo que la ecuacion para un voltaje alterno es:
\sub{$V_c=A\e{-tK}+\fraccion{5K}{K^2+\omega ^2}sin(\omega t)-\fraccion{5\omega}{K^2+\omega ^2}cos(\omega t)$}\\
\bigskip

\begin{figure}[htb]%h: here, t: top al principio de la pagina, b: al final de la página, !
		\centering %alineación 
		%%Figura*********************************************************
							%Simbologia europea o ameriaca, escala de la figura
		\begin{circuitikz}[american,scale=2.5, transform shape]
		    \draw (0,0) to [short,-*](0,0); 
			\draw (0,2) to [short,*-](0,2); 
			\draw (0,2) to [sV=$V$](0,0); 
			\draw (0,2) to [R,l=$R_{1}$,f=$i_{R_1}$](3.1,2); 
			\draw (3.1,2) to [C, l=$C$, f=$i_{C}$](3.1,0); 
			\draw (3.1,2)to[short,-](5.1,2);
			\draw  (5.1,2) to [R,l=$R_2$,f=$i_{R_{2}}$](5.1,0);
			\draw (5.1,0) to [short,-](0,0); 
		\end{circuitikz}
		\caption{RCSimple}
 		\label{RCSimple}
		%************************************************************Final Figura
\end{figure}
\bigskip
\end{document}